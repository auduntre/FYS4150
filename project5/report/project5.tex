\documentclass[a4paper, fontsize=11pt]{article}

\usepackage{amsmath,amsfonts,amsthm} % Math packages
\usepackage[english]{babel} % English language/hyphenation
%\usepackage{hyperref}
\usepackage{listings}
\usepackage{color}
\usepackage{verbatim}
\usepackage{graphicx}
\usepackage{float}
\usepackage{diagbox}
\usepackage{multirow}
\usepackage{subcaption}

\usepackage{pgfplotstable, booktabs, mathpazo}

\pgfplotstableset{
    every head row/.style={before row=\toprule,after row=\midrule},
    every last row/.style={after row=\bottomrule},
%    alias/{$T$}/.initial=0,
%    alias/{$E$}/.initial=1,
%    alias/{$C_V$}/.initial=2,
%    alias/{$M$}/.initial=3,
%    alias/{$chi$}/.initial=4,
%    alias/{$abs(M)$}/.initial=5,
}



\usepackage[colorlinks=true,linkcolor=black,urlcolor=blue,
citecolor=blue]{hyperref}

%\bibliographystyle{ieeetr}
\bibliographystyle{apalike}

\definecolor{dkgreen}{rgb}{0,0.6,0}
\definecolor{gray}{rgb}{0.5,0.5,0.5}
\definecolor{mauve}{rgb}{0.58,0,0.82}


\lstset{frame=tb,
  language=C++,
  aboveskip=3mm,
  belowskip=3mm,
  showstringspaces=false,
  columns=flexible,
  basicstyle={\small\ttfamily},
  numbers=none,
  numberstyle=\tiny\color{gray},
  keywordstyle=\color{blue},
  commentstyle=\color{dkgreen},
  stringstyle=\color{mauve},
  breaklines=true,
  breakatwhitespace=true,
  tabsize=3
}


\begin{document}

\title{Project 5 FYS4150 \\ Partial Differential equations}
\author{Audun Tahina Reitan \& Marius Holm}

%----------------------------------------------------------------------------------------
%	PROBLEM 1
%----------------------------------------------------------------------------------------
\maketitle


\section{Abstract}


\section{Introduction}

In this project we want to study the numerical stability and errors of the forward Euler, backward Euler, and Crank-Nicolson discretization schemes. In order to do so we study the diffusion equation 


\section{Methods}


\subsection{Discretization}

We wish to discretize the partial differential equation in order to create a system of linear equations which we can then solve using computationally. 

\paragraph{}
We start off by discretizing the time interval such that after $j$ time-steps the time is given by 

\begin{equation}
t_{j} = j \, \Delta t \quad j \geq 0
\end{equation}

We do the same for the spacial interval and split it into step lengths of the same size given by 

\begin{equation}
\Delta x = \dfrac{1}{n+1}
\end{equation}

such that the position after $i$ steps is given by 

\begin{equation}
x_{i} = i \, \Delta x \quad 0 \leq i \leq n+1
\end{equation}

More details regarding discretization of domains can be found in chapter 10 of the lecture notes. \cite{statphys} Below we apply three well-known discretization schemes to the diffusion equation

\begin{align}
\dfrac{\partial u(x,t)}{\partial t}  &= \dfrac{\partial^2 u(x,t)}{\partial x^2}, \quad t > 0, \: x \in [0, L]
\\
u_{t} &= u_{xx}
\end{align}

\subsubsection{Forward Euler}

We discretize the time dependent part of our differential equation according to the forward Euler scheme centered at time $t$.

\begin{equation}
u_{t} \approx \dfrac{u(x, t+ \Delta t) - u(x, t)}{\Delta t} = \dfrac{u(x_{i}, t_{j}+ \Delta t) - u(x_{i}, t_{j})}{\Delta t}
\end{equation}


\begin{align}
u_{xx} \approx& \dfrac{u(x + \Delta x, t) - 2u(x, t) + u(x - \Delta x, t)}{\Delta x^2}
\\
=&\dfrac{u(x_{i} + \Delta x, t_{j}) - 2u(x_{i}, t_{j}) + u(x_{i} - \Delta x, t_{j})}{\Delta x^2}
\end{align}

\subsubsection{Backward Euler}

\begin{equation}
u_{t} \approx \dfrac{u(x, t) - u(x, t - \Delta t)}{\Delta t} = \dfrac{u(x_{i}, t_{j}) - u(x_{i}, t_{j} -  \Delta t)}{\Delta t}
\end{equation}


\begin{align}
u_{xx} \approx& \dfrac{u(x + \Delta x, t) - 2u(x, t) + u(x - \Delta x, t)}{\Delta x^2}
\\
=&\dfrac{u(x_{i} + \Delta x, t_{j}) - 2u(x_{i}, t_{j}) + u(x_{i} - \Delta x, t_{j})}{\Delta x^2}
\end{align}

\subsubsection{Crank-Nicolson}

The implicit Crank-Nicolson scheme with a time-centered scheme at \\ $(x, t+\Delta t /2)$ as opposed to $(x, t)$ for the Euler schemes.

\begin{equation}
u_{t} \approx \dfrac{u(x, t+ \Delta t) - u(x, t)}{\Delta t} = \dfrac{u(x_{i}, t_{j}+ \Delta t) - u(x_{i}, t_{j})}{\Delta t},
\end{equation}

while the corresponding spatial second-order derivative is approximated as

\begin{align}
u_{xx} \approx &\dfrac{1}{2} \Bigg(\dfrac{u(x_{i} + \Delta x, t_{j}) - 2u(x_{i}, t_{j}) + u(x_{i} - \Delta x, t_{j})}{\Delta x^2} +
\\
&\dfrac{u(x_{i} + \Delta x, t_{j} + \Delta t) - 2u(x_{i}, t_{j}+ \Delta t) + u(x_{i} - \Delta x, t_{j}+ \Delta t)}{\Delta x^2}\Bigg)
\end{align}

\subsection{ }
\cite{statphys} \cite{IPDE}


\subsection{Implementation}



\paragraph{}
All our code, calculations, and plots used can be found in \href{https://github.com/MariusHolm/FYS4150}{my GitHub repository}. 

\section{Results}


\subsection{Analytic expressions}

\paragraph{}


\subsection{Numerical comparison}



%
%\begin{table}[h!tb]
%\centering
%\pgfplotstabletypeset[sci, precision = 5]{../code/src/opt_results/opt_nopara/nopara_2_1000_1.000000.dat}
%\caption{Numerical results for a system with $L=2$ with 1000 Monte Carlo cycles.}
%\label{l2_1000}
%\end{table}
%
%\begin{table}[h!tb]
%\centering
%\pgfplotstabletypeset[sci, precision = 5]{../code/src/opt_results/opt_nopara/nopara_2_10000_1.000000.dat}
%\caption{Numerical results for a system with $L=2$ with 10 000 Monte Carlo cycles.}
%\end{table}
%
%\begin{table}[h!tb]
%\centering
%\pgfplotstabletypeset[sci, precision = 5]{../code/src/opt_results/opt_nopara/nopara_2_100000_1.000000.dat}
%\caption{Numerical results for a system with $L=2$ with 100 000 Monte Carlo cycles.}
%\end{table}
%
%\begin{table}[h!tb]
%\centering
%\pgfplotstabletypeset[sci, precision = 5]{../code/src/opt_results/opt_nopara/nopara_2_1000000_1.000000.dat}
%\caption{Numerical results for a system with $L=2$ with 1 000 000 Monte Carlo cycles.}
%\end{table}
%
%\begin{table}[h!tb]
%\centering
%\pgfplotstabletypeset[sci, precision = 5]{../code/src/opt_results/opt_nopara/nopara_2_10000000_1.000000.dat}
%\caption{Numerical results for a system with $L=2$ with 10 000 000 Monte Carlo cycles.}
%\label{l2_10000000}
%\end{table}



\subsection{Most likely state}




%\begin{figure}[H]
%\begin{subfigure}[t]{0.5\linewidth}
%\includegraphics[scale=0.4]{../code/src/opt_results/opt_mpi/MCcycles/plots/t1plots/{Energy}.png}
%\end{subfigure}
%\begin{subfigure}[t]{0.5\linewidth}
%\includegraphics[scale=0.4]{../code/src/opt_results/opt_mpi/MCcycles/plots/t1plots/{AbsMagnetization}.png}
%\end{subfigure}
%\caption{Plots of the expecatation value for the energy and the absolute magnetization as functions of the number of Monte Carlo cycles. Using $T=1$. }
%\label{MCcycles1}
%\end{figure}


%\begin{figure}[H]
%	\begin{subfigure}[t]{0.5\linewidth}
%	\includegraphics[scale=0.4]{../code/src/opt_results/opt_mpi/MCcycles/plots/t24plots/{Energy}.png}
%	\end{subfigure}
%	\begin{subfigure}[t]{0.5\linewidth}
%	\includegraphics[scale=0.4]{../code/src/opt_results/opt_mpi/MCcycles/plots/t24plots/{AbsMagnetization}.png}
%	\end{subfigure}
%		\caption{Plots of the expecatation value for the energy and the absolute magnetization as functions of the number of Monte Carlo cycles. Here using $T=2.4$.}
%		\label{MCcycles2}
%\end{figure}



\subsection{Numerical studies of phase transitions}



\section{Conclusions}






\bibliography{references}
\end{document}