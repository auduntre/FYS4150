\documentclass[a4paper, fontsize=11pt]{article}

\usepackage{amsmath,amsfonts,amsthm} % Math packages
\usepackage[english]{babel} % English language/hyphenation
%\usepackage{hyperref}
\usepackage{listings}
\usepackage{color}
\usepackage{verbatim}
\usepackage{graphicx}
\usepackage{booktabs}
\usepackage{float}


\usepackage[colorlinks=true,linkcolor=black,urlcolor=blue,
citecolor=blue]{hyperref}

%\bibliographystyle{ieeetr}
\bibliographystyle{apalike}

\definecolor{dkgreen}{rgb}{0,0.6,0}
\definecolor{gray}{rgb}{0.5,0.5,0.5}
\definecolor{mauve}{rgb}{0.58,0,0.82}


\lstset{frame=tb,
  language=C++,
  aboveskip=3mm,
  belowskip=3mm,
  showstringspaces=false,
  columns=flexible,
  basicstyle={\small\ttfamily},
  numbers=none,
  numberstyle=\tiny\color{gray},
  keywordstyle=\color{blue},
  commentstyle=\color{dkgreen},
  stringstyle=\color{mauve},
  breaklines=true,
  breakatwhitespace=true,
  tabsize=3
}

\begin{document}

\title{Project 3 FYS4150 \\ Ordinary differential equations}
\author{Audun Tahina Reitan and Marius Holm}

%----------------------------------------------------------------------------------------
%	PROBLEM 1
%----------------------------------------------------------------------------------------
\maketitle


\section{Abstract}
We give a brief introduction to eigenvalue problems and how the Jacobi method can be used to solve such problems. We also give a brief discussion on scaling equations, and numerical stability for a relevant equation.

\paragraph{}
We then show that the dot product and orthogonality is preserved for an orthogonal transformation. We then present a small selection of results and two plots of relevance. More of our results can be found in the GitHub repository linked in the implementation section. 

\section{Introduction}
In this project we'll develop code for simulating the solar system, using the Verlet algorithm for solving coupled ordinary differential equations. For this project we will focus on the use of classes in our code, which makes it a lot easier to reuse larger parts of our code for problems of similar character. In our case of the solar system we can create a class describing a planet, a moon, or some other astronomical body which we want to include in our solar system.


\paragraph{}
In order to test that our algorithm works, we start by looking at a simple hypothetical system consisting of only the Earth orbiting the Sun. We then investigate the stability of our algorithm and plot the position of the Earth orbiting the Sun. Furthermore we consider the initial velocity needed for the planet to escape the gravity of the Sun. Then we'll consider a three-body system consisting of the Earth, the Sun, and Jupiter. We then expand our three-body model to the entire solar system including Pluto. In the final part of our project we look at the perihelion precession of Mercury.



\section{Methods}


\subsection{Differential equations}
The order of an ordinary differential equation (ODE) is given by the highest order of derivative found in the left-hand side of the equation. A first order differential equation is typically of the form 

\begin{equation}
\dfrac{dy}{dt} = f\big(t,\dfrac{dy}{dt},y\big)
\end{equation}

where $f$ is an arbitrary function. A well known second order differential equation is Newton's second law 

\begin{equation}
m \dfrac{d^2 x}{dt^2}=-kx
\end{equation}

where $k$ is the force constant. All ODE's depend on one variable only, compared to partial differential equations which can depend on several variables. PDE's are widely used, but won't be of any concern in this project.


\subsection{Newton's law of gravitation}


\subsection{Discretizing differential equations}


\subsection{Implementation}

.\cite{H-Jensen} 


\paragraph{}
All our code, calculations, and plots used can be found in \href{https://github.com/auduntre/FYS4150/tree/master/Project%203}{Auduns GitHub repository}.

\section{Results}
\subsection{Preservation of dot product}
We want to show that an orthogonal transformation preserves the dot product and orthogonality. Given a basis of vectors $\textbf{v}_{i}$

\begin{equation}
\textbf{v}_{i}=
\begin{bmatrix}
v_{i1} \\
v_{i2} \\
\vdots \\
v_{in}
\end{bmatrix}
\end{equation}

And assuming that the basis is orthogonal.

\begin{equation}
\textbf{v}^{T}_{j} \textbf{v}_{i} = \delta_{ij}
\end{equation}

The transformation is given by

\begin{equation}
\textbf{w}_{i} = \textbf{U} \textbf{v}_{i}
\end{equation}

As the transformation matrix \textbf{U} is orthogonal we have $\textbf{U}^{T}\textbf{U} = \textbf{I}$.

\begin{align*}
\textbf{w}_{i} \cdot \textbf{w}_{j} & =\textbf{U} \textbf{v}_{i} \cdot \textbf{U} \textbf{v}_{j} = (\textbf{U}\textbf{v}_{i})^{T}(\textbf{U}\textbf{v}_{j})=(\textbf{U}^{T} \textbf{v}_{i}^{T})(\textbf{U}\textbf{v}_{j})\\
&= \textbf{v}_{i}^{T} (\textbf{U}^{T} \textbf{U}) \textbf{v}_{j}= \textbf{v}_{i}^{T} \, \textbf{I} \, \textbf{v}_{j} =  \textbf{v}_{i}^{T} \,  \textbf{v}_{j} = \textbf{v}_{i} \cdot \textbf{v}_{j} = \textbf{v}^T_{i} \textbf{v}_{j}
\end{align*}

As $\textbf{w}_{i} \cdot \textbf{w}_{j} = \textbf{v}_{i} \cdot \textbf{v}_{j}$ the dot product is preserved. The same applies for $\textbf{v}_{i} \cdot \textbf{v}_{j} = \textbf{v}^T_{i} \textbf{v}_{j} = \delta_{ij}$ which implies that orthogonality is also preserved. This means that the Jacobi's rotational method which multiple orthogonal transfromations preserves the orthogonality and dot product of the columns in the system we transform.

\subsection{Eigenvalues of one electron in the harmonic oscillator}

Using one electron in the harmonic oscillator and solving using the Jacobi's method we get the following result on the apporixmation of the analytical eigenvalues $\lambda = 3, 7, 11, 15$ for the system, varying the number of integration points $N$:

\begin{table}[htp]
\begin{center}
\begin{tabular}{@{}llllllll@{}}
\toprule
     & \multicolumn{7}{c}{N}               \\ \cmidrule(l){2-8} 
Absolute error & 10 & 25 & 50 & 75 & 100 & 150 & 200 \\ \midrule
$|\lambda_1 - \hat{\lambda}_1|$    &    $8.05\mathrm{E}{-2}$&    $1.25\mathrm{E}{-2}$&    $3.13\mathrm{E}{-3}$&    $1.39\mathrm{E}{-3}$&     $7.81\mathrm{E}{-4}$&     $3.47\mathrm{E}{-4}$&      $1.95\mathrm{E}{-4}$\\
$|\lambda_2 - \hat{\lambda}_2|$    &    $4.16\mathrm{E}{-2}$&    $6.30\mathrm{E}{-2}$&    $1.57\mathrm{E}{-2}$&    $6.95\mathrm{E}{-3}$&     $3.91\mathrm{E}{-3}$&     $1.73\mathrm{E}{-3}$&      $9.74\mathrm{E}{-4}$\\
$|\lambda_3 - \hat{\lambda}_3|$    &    $1.06$&    $1.54\mathrm{E}{-1}$&    $3.81\mathrm{E}{-2}$&    $1.68\mathrm{E}{-2}$&     $9.34\mathrm{E}{-3}$&     $4.04\mathrm{E}{-3}$&     $2.18\mathrm{E}{-3}$\\
$|\lambda_4 - \hat{\lambda}_4|$    &    $2.08$&    $2.83\mathrm{E}{-1}$&    $6.54\mathrm{E}{-2}$&    $2.57\mathrm{E}{-2}$&     $1.18\mathrm{E}{-2}$&     $2.02\mathrm{E}{-3}$&     $1.42\mathrm{E}{-3}$\\ \bottomrule
\end{tabular}
\caption{Absolute error between the analytically known eigenvalues of the one-electron energies and our numerically calculated eigenvalues.}
\end{center}
\end{table}


\subsection{Quantum dots, plots}


%\begin{figure}[H]
%\includegraphics[scale=0.8]{}
%\caption{Comparing the energy levels for a system with and without Coulomb interation with $\omega_{r}=0.01$.}
%\label{Coulomb}
%\end{figure}


%\begin{figure}
%\includegraphics[scale=0.8]{}
%\caption{Probability distribution for the ground state of a two-electron system including Coulomb interaction. Plotted for different values of $\omega_{r}$, where $\omega_{r}$ is a parameter which reflects the strength of the oscillator potential.}
%\label{two-electron}
%\end{figure}



\section{Discussion}
\subsection{Jacobi method}
The Jacobi method is generally a slower algorithm than those based tridiagonalization. However the Jacobi method is easy to parallelize which can improve performance significantly. The main reason for the Jacobi methods slow convergence is that for each new rotation, matrix elements which were zero might change to non-zero values.


\paragraph{}
For the case with one electron we would have needed to choose $N > 200$ in order to get numerical eigenvalues, $\hat{\lambda}_i$ that are able to reproduce the analytical ones, $\lambda_i$ with four leading digits after the decimal point (which means an absolute error $< 10^{-4}$). The Jacobi method is very slow for too large $N$ values, and doing the calculations with an $N > 200$ a modern laptop takes several minutes to complete the calculations. 

\paragraph{}
It would be of interest to investigate more effective methods like the Householder's method for large values of $N$, and see what differences a significant increase in integration points would result in.


\bibliography{references}
\end{document}